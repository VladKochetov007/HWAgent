
\documentclass{article}
\usepackage{amsmath}
\usepackage{listings}
\usepackage{xcolor}

\definecolor{codegreen}{rgb}{0,0.6,0}
\definecolor{codegray}{rgb}{0.5,0.5,0.5}
\definecolor{codesilver}{rgb}{0.9,0.9,0.9}
\definecolor{backcolour}{rgb}{0.98,0.98,0.98}

\lstdefinestyle{mystyle}{
    backgroundcolor=\color{backcolour},
    commentstyle=\color{codegreen},
    keywordstyle=\color{magenta},
    numberstyle=\tiny\color{codegray},
    stringstyle=\color{purple},
    basicstyle=\ttfamily\footnotesize,
    breakatwhitespace=false,
    breaklines=true,
    captionpos=b,
    keepspaces=true,
    numbers=left,
    numbersep=5pt,
    showspaces=false,
    showstringspaces=false,
    showtabs=false,
    tabsize=2
}

\lstset{style=mystyle}

\title{Factorial Function Implementation and Verification}
\author{}
\date{}

\begin{document}

\maketitle

\section*{Problem Description}
The task is to write a simple Python function to calculate the factorial of a non-negative integer.

\section*{Solution Approach}
A Python function named `factorial(n)` will be implemented. This function will:
\begin{itemize}
    \item Check if the input `n` is a non-negative integer. If not, it will raise a `ValueError`.
    \item Handle the base case where $n=0$, returning 1.
    \item For $n > 0$, calculate the factorial iteratively by multiplying numbers from 1 to $n$.
\end{itemize}

\section*{Python Code Implementation}
The following Python code implements the `factorial` function:

\lstinputlisting[language=Python]{factorial_function.py}

\section*{Verification}
To verify the correctness of the function, several test cases were executed, including:
\begin{itemize}
    \item Factorial of 0
    \item Factorial of 1
    \item Factorial of 5
    \item Factorial of 10
    \item Invalid inputs (negative number and non-integer) to check error handling.
\end{itemize}

The Python script was executed, and the output is shown below:

\begin{verbatim}
Factorial of 0: 1
Factorial of 1: 1
Factorial of 5: 120
Factorial of 10: 3628800
Error for -3: Input must be a non-negative integer.
Error for 3.5: Input must be a non-negative integer.
\end{verbatim}

\section*{Conclusion}
The implemented `factorial` function correctly calculates the factorial for non-negative integers and handles invalid inputs as specified. The verification steps confirm the function's accuracy and robustness.

\end{document}
