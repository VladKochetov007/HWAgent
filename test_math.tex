\documentclass[12pt,a4paper]{article}

\usepackage[utf8]{inputenc}
\usepackage[russian,english]{babel}
\usepackage{amsmath,amssymb,amsfonts}
\usepackage{geometry}
\usepackage{hyperref}
\usepackage{graphicx}
\usepackage{xcolor}
\usepackage{listings}
\usepackage{algorithm}
\usepackage{algorithmic}

\geometry{a4paper, margin=1in}

% Code styling
\definecolor{codegray}{rgb}{0.5,0.5,0.5}
\definecolor{codepurple}{rgb}{0.58,0,0.82}
\definecolor{backcolour}{rgb}{0.95,0.95,0.92}

\lstdefinestyle{mystyle}{
    backgroundcolor=\color{backcolour},   
    commentstyle=\color{codegray},
    keywordstyle=\color{blue},
    numberstyle=\tiny\color{codegray},
    stringstyle=\color{codepurple},
    basicstyle=\ttfamily\footnotesize,
    breakatwhitespace=false,         
    breaklines=true,                 
    captionpos=b,                    
    keepspaces=true,                 
    numbers=left,                    
    numbersep=5pt,                  
    showspaces=false,                
    showstringspaces=false,
    showtabs=false,                  
    tabsize=2
}

\lstset{style=mystyle}


\title{Вычисление Интеграла}
\author{AI Technical Assistant}
\date{\today}

\begin{document}
\maketitle
\tableofcontents
\newpage

\section{Постановка задачи}
Требуется вычислить интеграл \(\int (x^2 + 1) dx\)

\section{Методология}
Используем правило степени для интегрирования

\section{Пошаговое решение}
\[\int (x^2 + 1) dx = \int x^2 dx + \int 1 dx = \frac{x^3}{3} + x + C\]

\section{Вычислительная проверка}
Проверка с помощью Python и sympy

\section{Анализ результатов}
Результат соответствует аналитическому решению

\section{Заключение}
Интеграл успешно вычислен

\end{document}
